\input ../../../../include/include.tex

\begin{document}

\velkynadpis{Výber oblečenia}

% proofread by Sameth
Vyberať si oblečenie vôbec nie je jednoduché. Spýtajte sa ľubovoľného dievčaťa. Predsa hráškovo
ružová sa nehodí k hrozienkovo zlatej a tyrkysová nemôže ísť dokopy so slonovinovo-maslovou čiernou
\footnote{Nebojte sa, žiadna z predchádzajúcich farieb neexistuje, aj keď vám dievčatá budú tvrdiť
opak}. A takisto kárované nohavice nemôžu ísť dokopy s pásikavým tričkom. A to je len začiatok.

Pipka Natálka to však našťastie má zmáknuté a presne vie, ktoré veci sa k sebe hodia. Stojí teda v
obchode a prezerá si všetkých $n$ kusov oblečenia čo tam majú a rozmýšľa, ktoré tri si vyberie a
kúpi. Chce byť však šporovlivá a utratiť čo najmenej peňazí, ale zároveň sa každá z tých vecí musí
hodiť k zvyšným dvom. Pomôžte jej to zistiť.

\nadpis{Úloha}

Na vstupe dostanete ceny $n$ vecí v obchode a takisto $m$ párov oblečení, ktoré sa k sebe hodia.
Uvedomte si, že ak sa vec $a$ hodí k $b$, tak sa aj $b$ hodí k $a$. Zistite, koľko najmenej peňazí
potrebuje, ak si chce kúpiť práve tri veci, ktoré sa k sebe navzájom hodia.

\nadpis{Vstup}

Na prvom riadku sú dve celé čísla $n$ a $m$ ($3 \leq n \leq 100$, $0 \leq m \leq \frac{n(n-1)}{2}$)
-- počet oblečenia v obchode a počet párov, ktoré sa k sebe hodia.

Na druhom riadku je $n$ celých čísel $a_i$ ($1 \leq a_i \leq 10^6$) -- ceny jednotlivých kusov oblečenia.

Posledných $m$ riadkov obsahuje dvojice $x_i, y_i$ -- pár oblečenia, ktorý sa k sebe hodia. Oblečenia
sú číslované od $1$ po $n$.

\nadpis{Výstup}

Vypíšte najmenšiu sumu takú, že sa za ňu dajú kúpiť tri kusy oblečenia, ktoré sa k sebe hodia, alebo
vypíšte $-1$ ak nájsť takú trojicu vecí nie je možné.

\nadpis{Príklady}

\vstup
3 3
1 2 3
1 2
2 3
3 1
\vystup
6
\koniec

\vstup
3 2
2 3 4
2 3
2 1
\vystup
-1
\koniec

\end{document}
