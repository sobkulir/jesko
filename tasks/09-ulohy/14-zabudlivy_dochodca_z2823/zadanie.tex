\input ../../../include/include.tex

\begin{document}

\velkynadpis{Zbláznil sa fyzikár\dots}
\zadanie{Zábudlivý dôchodca}
V~dobe krutej finančnej krízy sa snaží každá továreň minimalizovať svoje náklady na prevádzku,
preto sa prijímajú aj lacnejšie pracovné sily z~radov dôchodcov, študentov a~nelegálnych prisťahovalcov.
Inak tomu samozrejme nie je ani v~továrni na triedenie čísiel. Preto keď prišla ponuka pána Michala, že
bude otročiť za jedlo, šaty a~vzduch, bolo vedenie nadšené.

Rýchlo ale zistili, že to má drobný háčik. Pán
Michal škúli, takže nikdy nevidí dve susedné čísla. To by ešte nebolo také hrozné, kebyže neprišiel pri
absurdnej nehode s~kapustou a~zbíjačkou o~krátkodobú pamäť. Takže príslovie {\sl zíde z~očí,
zíde z~mysle} preňho bohužiaľ platí doslovne.

Majiteľ však zistil, že keď nechá pána Michala triediť niektoré
zákazky, napriek svojmu handicapu ich utriediť dokáže. Chcel by mať program, ktorý by mu pri zadaní zákazky
povedal, či ju môže dať na starosť pánovi Michalovi. No a~keď už je tá kríza, a~vy ste študenti\dots môžete hádať.

\uloha
Na vstupe máte dané číslo $N$ ($1 \leq N \leq 100000$), v~ďalšom riadku nasleduje $N$ celých čísiel (menej ako $1000000$). Vašou úlohou je rozhodnúť,
či sme schopní získať z~týchto čísiel neklesajúcu postupnosť, ak v~jednom kroku môžeme
vymeniť iba dvojicu čísel na pozíciách $k$ a~$k+2$ ($1 \leq k \leq N - 2$).

\priklad
\vstup\tt\obeylines
5
10 45 52 14 37

\vystup\tt\obeylines
Ano.

{\sl Vymeníme najprv čísla na treťom a~piatom mieste, potom tie na druhom a~štvrtom mieste.}
\koniec



\priklad
\vstup\tt\obeylines
4
1 3 2 4

\vystup\tt\obeylines
Nie.

{\sl Výmenou prvého a~tretieho, ani výmenou druhého a~štvrtého čísla si veľmi nepomôžeme.}
\koniec

\end{document}
