\input ../../../include/include.tex

\begin{document}

\velkynadpis{Zbláznil sa fyzikár\dots}

Váš učiteľ fyziky sa rozhodol vyskúšať nové pedagogické metódy. Už sa na
hodinách skoro nič neučíte, len stále experimentujete a niečo meriate. To mu
nemožno až tak zazlievať, aspoň vidíte fyzikálne deje v praxi. No napriek tomu
všetci spolužiaci frflú, že sa azda pomiatol, veď zadáva jednu domácu úlohu za
druhou.

Posledný mesiac ste strávili neustálym spisovaním protokolov z laboratórnych
cvičení. Bolo toho tak veľa, že sa to jednoducho nedalo stíhať -- mnoho z nich
ste dopisovali narýchlo, tesne pred hodinou. A tak nečudo, že fyzikár nebol s
kvalitou spokojný. Údaje nepresné, výpočty chýbajú, tabuľky len krivo rukou
načmárané, namiesto aby boli poriadne narysované.

Verdikt: prepracovať! No ešte to tak chýbalo! Čo teraz? Nechce sa vám predsa
všetko robiť odznova. Fajn, obsah môžete odpísať od svedomitejších spolužiakov,
ale tie tabuľky\dots tie bude treba narysovať, jednu za druhou. Desiatky
tabuliek\dots nezáživná, zdĺhavá, mechanická práca\dots priam práca pre robota,
nie človeka.  Teda, robota alebo\dots počítač! Pri tejto spásonosnej myšlienke
sa vám uľavilo.  Ešteže viete programovať!

\uloha
Napíšte program, ktorý na vstupe dostane dve prirodzené čísla $R$ a $S$ a
vykreslí tabuľku s $R$ riadkami a $S$ stĺpcami. Na vodorovné línie použite znak
'{\tt -}' (mínus, resp. pomlčka), na zvislé línie znak '{\tt |}' (tzv. rúra) a
na ich priesečníky znak '{\tt +}' (plus). Každé políčko tabuľky je tvorené
jedinou medzerou tesne ohraničenou vertikálnymi a horizontálnymi líniami.

\napdis{Formát vstupu}

V prvom riadku budú dve celé čísla $R$ a $S$ ($1 <= R, S <= 100$), oddelené 
medzerou.


\napdis{Formát výstupu}

Na výstup vypíšte požadovanú tabuľku.

\priklad
\vstup\tt\obeylines
2 3

\vystup\tt\obeylines
+-+-+-+
| | | |
+-+-+-+
| | | |
+-+-+-+

\koniec

\end{document}
