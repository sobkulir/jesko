\input ../../../include/include.tex

\begin{document}

\velkynadpis{Zázračná opička}

Marek sa rozhodol, že svoju zázračnú opičku\footnote{napríklad dokáže skákať na ľubovoľne veľké
vzdialenosti} naučí nový trik. Najprv rozložil niekoľko stoličiek do kruhu
a na jednu zo stoličiek posadil opičku. Potom začal tlieskať.

Pri každom tlesknutí opička skočí o niekoľko miest proti smeru hodinových ručičiek. Pri prvom tlesknutí skočí na susednú
stoličku. Pri druhom skočí z tej, na ktorej práve sedí, o dve stoličky ďalej. Pri treťom tlesknutí
skočí o tri stoličky, \dots{} Takto to pokračuje
donekonečna: pri každom ďalšom tlesknutí spraví opička o 1 dlhší skok ako pred tým.

Mareka by teraz zaujímalo, na koľko rôznych stoličiek opička vôbec niekedy doskočí.

\nadpis{Úloha}

V kruhu je $n$ stoličiek a na jednej z nich sedí opička. Opička následne spraví
nekonečne veľa skokov v tom istom smere, pričom $k$-ty skok je dlhý $k$-stoličiek.

Teda ak si stoličky očíslujeme 0 až $n-1$ proti smeru ručičiek, tak platí:
Ak pred $k$-tym skokom bola opička na $i$-tej stoličke, tak po tomto skoku bude
na stoličke číslo ($(k+i) \bmod n$), kde $a\bmod b$ je zvyšok, ktorý dáva $a$ po delení $b$.

Spočítajte, koľko je takých stoličiek, že na ne opička niekedy skočí.

\nadpis{Formát vstupu}

Na vstupe je jedno celé číslo $n$ udávajúce počet stoličiek. $1\leq n\leq 1\,000\,000$.

\nadpis{Formát výstupu}

Vypíšte jeden riadok a v ňom jedno číslo: na koľko rôznych stoličiek opička niekedy skočí.


\nadpis{Príklady}

\vstup
3
\vystup
2
\koniec

{\sl Začne na stoličke 0, skočí na stoličku 1, odtiaľ na stoličku 0, odtiaľ znova na stoličku 0
(skokom o 3 miesta), odtiaľ na stoličku 1, \dots\ Dá sa dokázať, že na stoličku 2 neskočí nikdy.}

\vstup
4
\vystup
4
\koniec

\end{document}
