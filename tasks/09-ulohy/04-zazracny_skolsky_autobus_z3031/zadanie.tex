\input ../../../include/include.tex

\begin{document}

\velkynadpis{Zázračný školský autobus}


Možno poznáte rozprávku s názvom Zázračný školský autobus.
Je to rozprávka o deťoch, ktoré poznávajú svet pomocou zázračného autobusu,
ktorý vie lietať, plávať, zmeniť sa na raketu\dots\ Vie urobiť vlastne všetko,
čo si len zmyslíte. Verte či nie, tento autobus naozaj existuje
a svoju zázračnú neviditeľnú garáž má priamo v~Bratislave.
Maru ho občas vidí zo svojej izby ako vlietava alebo vylieta zo svojej garáže.
Keď má šťastie a má pri sebe fotoaparát, tak ho odfotí.
To sa stáva len zriedka, lebo autobus väčšinou rýchlo zmizne.

Teraz má Maru doma hŕbu fotiek zázračného autobusu. Rozhodla sa, že si urobí štatistiku,
koľkokrát ho odfotila, keď vylietaval a koľkokrát, keď vlietaval do garáže.
Tých fotiek je však strašne veľa a Maru ako vždy nemá čas. Preto vás žiada o pomoc.
Pomôžte Maru rozhodnúť, či autobus na fotke vlietaval, alebo vylietaval z~garáže.
Bude stačiť ak zistíte, či letel doprava alebo doľava.

\nadpis{Úloha}

\newcommand{\bs}{\symbol{`\\}}
\newcommand{\us}{\symbol{`_}}

Na vstupe bude obrázok autobusu zložený zo znakov
\texttt{/}, \texttt{|}, \texttt{\bs}, \texttt{\us}, \texttt{-}, \texttt{O}, \texttt{o}, \texttt{.}  
a \texttt{H}.
Vašou úlohou je zistiť, či autobus na obrázku letí doprava alebo doľava.
Môžete si byť istí, že letí vodorovne a je otočený bokom k vám.

\nadpis{Formát vstupu}

V prvom riadku vstupu budú dve celé čísla $r$ a $s$ $(1 \leq r,s \leq 1\,000)$
udávajúce počet riadkov a písmen v~riadkoch, ktoré tvoria obrázok autobusu.
Ďalej nasleduje $r$ riadkov. V každom riadku je presne $s$ znakov.
Znaky, ktoré môžu byť v obrázku, znamenajú toto: \texttt{/}, \texttt{|}, \texttt{\bs},
\texttt{\us}, \texttt{-}  sú hrany karosérie, \texttt{.} je plocha karosérie alebo pozadie,
\texttt{O} je koleso, \texttt{o} je okno a \texttt{H} sú dvere.
Žiaden iný znak sa tam nevyskytne.

\nadpis{Formát výstupu}

Vypíšte slovo \texttt{doprava}, ak autobus letí doprava, prípadne slovo \texttt{dolava}, ak letí doľava.

\nadpis{Príklady}


%...............______________.............
%...........___/....___/......\............
%.......___/......./.........oo\...........
%....._/........__/....oo..ooooo\..........
%..../....___---...\.ooooo..oooo.\.........
%.../__---..oo..HHH.\.ooooo..oo...----__...
%../.o..ooo.ooo..HHH.\.oo............|||\..
%..\.oo.oooo.ooo..HHH.\..................|.
%...\.oo..o........HHH.\..___.|||.....___|.
%....\..............HHH.\/OOO\.....__/OO...
%.....\..............HHH.\OOOO\___/..OO....
%......\...________-------.OOOOOO..........
%.......---.OOO.............OOOO...........
%..........................................
\vstup
14 42
...............______________.............
...........___/....___/......\bs{}............
.......___/......./.........oo\bs{}...........
....._/........__/....oo..ooooo\bs{}..........
..../....___---...\bs{}.ooooo..oooo.\bs{}.........
.../__---..oo..HHH.\bs{}.ooooo..oo...----__...
../.o..ooo.ooo..HHH.\bs{}.oo............|||\bs{}..
..\bs{}.oo.oooo.ooo..HHH.\bs{}..................|.
...\bs{}.oo..o........HHH.\bs{}..___.|||.....___|.
....\bs{}..............HHH.\bs{}/OOO\bs{}.....__/OO...
.....\bs{}..............HHH.\bs{}OOOO\bs{}___/..OO....
......\bs{}...________-------.OOOOOO..........
.......---.OOO.............OOOO...........
..........................................
\vystup
doprava
\komentar
\includegraphics[width=230pt]{prikl1-autobus.jpg}
\koniec
%..............|..........
%......._____|...|_____...
%.._.../.oo.oo.oo.HH.oo\..
%.| \_/..oo.oo.oo.HH.ooo\.
%.| |_............HH....|.
%.|_/.\_________________/
%.........................
\vstup
7 25
..............|..........
......._____|...|_____...
.._.../.oo.oo.oo.HH.oo\bs{}..
.|.\bs{}_/..oo.oo.oo.HH.ooo\bs{}.
.|.|_............HH....|.
.|_/.\bs{}_________________/.
.........................
\vystup
doprava
\komentar
Zázračný školský autobus sa vie premeniť napríklad aj na ponorku.
\koniec

\end{document}
