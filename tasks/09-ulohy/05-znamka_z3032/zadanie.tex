\input ../../../include/include.tex

\begin{document}

\velkynadpis{Známka}

Žaba na Vianoce dostal stroj času, a tak sa ho ešte cez prázdniny rozhodol vyskúšať. Keďže má vyššie ciele
ako napríklad zabiť Hitlera, premiestnil sa do času pred štyrmi rokmi, keď ešte usilovne študoval na gymnáziu v slávnom
Lučenci.

Tam sa úspešne vlámal do budovy školy s jedinou myšlienkou v hlave: nájsť klasifikačný hárok a prepísať
niekoľko známok v ňom tak, aby mu na polrok vychádzala z nemčiny "čistá dvojka" -- teda aby jeho priemer bol
presne $2.0$. Tak totiž docieli, že vtedajší Žaba dostane bez ďalšieho skúšania dvojku. (No a to, aby sa vtedajší
Žaba vyhol skúšaniu a pritom dostal známku, ktorá je dosť dobrá, no nie až podozrivo dobrá, je veľmi podstatné
pre správny chod vesmíru.)

\nadpis{Úloha}

Na vstupe dostanete počet jednotiek, dvojok, trojok, štvoriek a pätiek v klasifikačnom hárku. Zistite,
koľko najmenej z nich treba zmeniť (t.j. prepísať na ľubovoľnú známku), aby ich priemer bol presne $2$.

\nadpis{Formát vstupu}

V jedinom riadku vstupu je päť čísel:  $z_1,\ldots,z_5$, pričom $z_i$ je počet známok s hodnotou $i$, ktoré má Žaba
v klasifikačnom hárku. Súčet počtov známok neprekročí $10^{10}$.

\textbf{Pozor!} Klasické premenné, \verb!longint! v Pascale a \verb!int! v C++, majú obmedzenú
veľkosť, čiže sa do nich zmestia čísla len od $-2^{31}$ do $2^{31}-1$ ($-2\,147\,483\,648,\dots,
2\,147\,483\,647$) vrátane. Čísla na vstupe môžu presiahnuť túto veľkosť. Použite
\verb!int64! resp. \verb!long long!.

\nadpis{Formát výstupu}

Vypíšte jeden riadok a v ňom jedno celé číslo: najmenší počet známok, po zmene ktorých bude Žabov priemer
presne $2.0$.

\nadpis{Príklady}% alebo Príklady, ak ich je viac

\vstup
1 2 1 0 2
\vystup
2
\komentar
Priemer je pred zmenou $3.0$. Na priemer $2.0$ stačí zmeniť napríklad obe päťky na dvojky.
\koniec

\vstup
3 1 0 0 0
\vystup
1
\komentar
Priemer je pred zmenou $1.25$. Priemer $2.0$ sa dá dosiahnuť napríklad zmenou jednej jednotky na štvorku.
\koniec

\end{document}
