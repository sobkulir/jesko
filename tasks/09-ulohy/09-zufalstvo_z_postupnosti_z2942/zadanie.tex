\input ../../../include/include.tex

\begin{document}

\velkynadpis{Zúfalstvo z postupností}

Marek Ospalý, Lin Jahodová, Miško Zelený a jeho princeznička slečna L. sú na
dovolenke na horách.

Každý večer si rozložia mapu a rozhodujú sa, na ktorý kopec sa vydajú
nasledujúci deň. Lenže také vyberanie túry nie je jednoduchá záležitosť, teda
aspoň nie pre tamtú štvoricu. Fascinujú ich totiž číselné postupnosti a podľa
nich si aj vyberajú túru. Našťastie majú so sebou instantného Janíčka v
orákulu a on im vie povedať postupnosť nadmorských výšok po každom metri trasy.

Miško je viac informatikom ako matematikom, preto sa mu páčia aj aritmetické, aj
geometrické postupnosti. Slečna L. je viac matematičkou ako informatičkou
(zatiaľ), preto sa jej páčia geometrické postupnosti. Lin
je opica\footnote {\url{https://plus.google.com/116535840236499706820/posts}} a má
veľmi úzky okruh obľúbených postupností. Aby sa jej postupnosť páčila, musí byť
aritmetická a zároveň aj geometrická. Marek o sebe tvrdí, že si názor na postupnosti
ešte nevytvoril, preto mu nie je dobrá žiadna.\footnote{To je len zámienka.
V skutočnosti je lenivý a nikam sa mu nechce chodiť.}

\nadpis{Úloha}

Rozpoznávať postupnosti je vcelku nudná činnosť. Našej štvorici výletníkov by sa
preto zišiel program, ktorý pre Janíčkovu postupnosť nadmorských výšok vypíše
zoznam účastníkov takejto túry.

Postupnosť je aritmetická, ak sa dá zapísať v tvare $a, a+d, a+2d, \dots,
a+(n-1)d$ pre nejaké $a, d \in \mathbb{R}$.

Postupnosť je geometrická, ak sa dá zapísať v tvare $a, aq, aq^2, aq^3, \dots,
aq^{n-1}$ pre nejaké $a, q \in \mathbb{R}$.

\nadpis{Formát vstupu}

Na prvom riadku je číslo $1 < n < 10^{6}$, určujúce počet prvkov postupnosti. Na druhom riadku nasleduje $n$ čísel oddelených medzerami. Prvky sú celé
čísla, ktoré sa zmestia do 32-bitovej premennej so znamienkom.

\nadpis{Formát výstupu}

Na prvý riadok vypíšte slovo "ano"/"nie" podľa toho, či je postupnosť aritmetická.
Na druhy riadok vypíšte slovo "ano"/"nie" podľa toho, či je postupnosť geometrická.

\nadpis{Príklady}

\vstup
6
-8 1 1000 32 238 -47
\vystup
nie
nie
\koniec
\textsl{Postupnosť nie je aritmetická ani geometrická. Taký výlet by sa nikomu nepáčil.}

\vstup
7
5 8 11 14 17 20 23
\vystup
ano
nie
\koniec
\textsl{Aritmetická postupnosť, za sebou idúce členy sa od seba líšia o 3. Páči sa iba Miškovi.}

\vstup
3
8 24 72
\vystup
nie
ano
\koniec
\textsl{Geometrická postupnosť, za sebou idúce členy sa líšia trojnásobne. Lin Jahodovej sa nepáči, takže Miško so slečnou L. budú mať romantický výlet :-)}

\vstup
4
99 0 0 0
\vystup
nie
ano
\koniec
\textsl{Táto postupnosť je geometrická, pretože jej členy môžeme zapísať ako $99, 99\cdot{}0, 99\cdot{}0^{2}, 99\cdot{}0^{3}$. Páči sa Miškovi a slečne L.}

\end{document}