\input ../../../include/include.tex

\begin{document}

\velkynadpis{Banková lúpež}

% proofread by Sameth (nenasiel som chybu)
Ježko Dušan a čajka Karolínka sú zlodeji. A špecializujú sa na banky. Karolínka vždy najskôr zvedie
riaditeľa banky a po prehýrenej noci sa vytratí z jeho bytu aj s kľúčom od trezoru. Na ďalší deň
Dušan nabehne zamaskovaný do banky a vďaka kľúču ju bleskovo vylúpi a rýchlo zmizne.

Bol dohodnutý s Karolínkou, že jej nechá jej podiel v smetnom koši na rohu Smrekovej a Bažinovej.
Dušan je však ježko nepoctivý a rozhodol sa čajku ošidiť a dať jej menší podiel. Z banky ukradol $n$
mincí hodnôt $a_1$ až $a_n$ (niektoré môžu byť aj rovnaké). Teraz by si chcel zobrať väčší podiel.
To znamená, že súčet hodnôt mincí, ktoré si zoberie Dušan musí byť \textbf{väčší} ako súčet mincí,
ktoré dá Karolínke.

Aby však nebol príliš nápadný, chce si zobrať mincí čo najmenej. Zistite koľko najmenej mincí si
Dušan musí zobrať, aby ich súčet bol väčší ako súčet zvyšných mincí.

\nadpis{Úloha}

Na vstupe dostanete hodnoty mincí, ktoré Dušan ukradol. Zistite, koľko najmenej mincí si musí
zobrať, aby mal \textbf{väčší} podiel.

\nadpis{Vstup}

Na prvom riadku je číslo $n$ ($1 \leq n \leq 1000$) -- počet ukradnutých mincí.

Na druhom riadku sú hodnoty $a_1$ až $a_n$ ($1 \leq a_i \leq 1000$).

\nadpis{Výstup}

Vypíšte jedno celé číslo -- najmenší počet mincí, ktoré si Dušan musí zobrať, aby mal väčší podiel.

\nadpis{Príklady}

\vstup
2
8 8
\vystup
2
\komentar
Dušan si musí zobrať obe mince. Ak by si zobral len jednu, podiel by bol rovnaký.
\koniec

\vstup
6
4 3 6 3 3 2
\vystup
3
\komentar
Ak si Dušan zoberie mince $6$, $3$ a $3$ bude mať v súčte $12$ peňazí, kým Karolínke zostane len
$9$.
\koniec

\end{document}
