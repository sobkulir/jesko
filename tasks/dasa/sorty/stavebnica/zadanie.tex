\input ../../../../include/include.tex

\begin{document}

\velkynadpis{Stavebnicové kocky}

Ježko Vladko má hromadu stavebnicových kociek rôznych veľkostí, usporiadané pekne od najmenšej po
najväčšiu. Rád by daroval nejakú sadu týchto kociek svojej kamarátke kačičke Márii. Mária je však
trošku náročná a nie je ochotná prijmuť ľubovoľnú sadu kociek -- predsa len sa jej musia hodiť k
topánkam. Našťastie, nie je ani tak rozmaznaná a je ochotná prijmuť viacero sád. Vladko vie, aké
sadt kociek Mária prijme. Pomôžte mu zistiť, ktoré z týchto sád sa nachádzahú medzi jeho kockami.

\nadpis{Úloha}

Na vstupe dostanete popis Vladkových kociek, usporiadaných podľa veľkosti. Taktiež budete mať zoznam
sád, ktoré Mária akceptuje. Každá sada sa skladá z niekoľkých kociek a je usporiadaná podľa veľkosti.
Pre každú sadu zistite, či sa dá vyskladať z Vladkových kociek, teda či má Vladko všetky kocky v
danej sade.

\nadpis{Vstup}

Na prvom riadku sú dve celé čísla $n$ a $q$ ($1 \leq n \leq 100000$, $1 \leq q \leq 1000$) -- počet
Vladkových kociek a počet dobrých sád.

Na ďalšom riadku je $n$ čísiel $a_1$ až $a_n$ určujúce veľkosti Vladkových kociek. Platí, že $a_1
\leq a_2 \leq \dots \leq a_n$.

Nasleduje $q$ riadkov, každý obsahuje popis jednej sady kociek, ktoré Mária prijme. Každý riadok
začína celým číslom $k_i$ ($1 \leq k_i \leq 1000$), ktoré určuje počet kociek v danej sade. Nasleduje $k_i$ čísiel
$b_{i,1}$  až $b_{i, k_i}$ -- veľkosti kociek v danej sade. Platí, že $b_{i,1} \leq b_{i,2} \leq \dots
\leq b_{i,k_i}$.

Môžete predpokladať, že súčet všetkých $k_i$ nepresiahne $10000$. Taktiež, všetky $a_i$ a $b_{i,j}$
sú menšie ako $10^9$.

\nadpis{Výstup}

Vypíšte $q$ riadkov, riadok $i$ obsahuje odpoveď na $i$-tu sadu kociek. Ak sa z Vladkových kociek dá
vybrať daná sada, vypíšte slovo \texttt{ano}, inak vypíšte \texttt{nie}.

\nadpis{Príklady}

\vstup
10 4
1 3 6 7 7 13 14 20 21 100
3 1 6 13
4 1 2 3 6
5 3 6 7 7 100
4 6 7 7 7
\vystup
ano
nie
ano
nie
\komentar
Druhá sada: Vladkove kocky neobsahujú kocku $2$.

Štvrtá sada: Vladkove kocky obsahujú kocku $7$ len dvakrát.
\koniec

\end{document}
