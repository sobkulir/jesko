\input ../../../include/include.tex

\begin{document}

\velkynadpis{Výstavný kus}

Ježko si nazbieral veľa húb. A jedna z nich je naozaj parádny výstavný kus. Má ich teraz rozložené
pred sebou na poličke. Chcel by však, aby jeho výstavný kus bol na hrdej prvej pozícii. Pomôžte mu
to spraviť.

\nadpis{Úloha}

Na vstupe dostanete popis poličky, teda poradie v akom sú huby uložené a tiež nakoľko sú červivé.
Samozrejme, najmenej červivá huba je ježkov výstavný kus. Chce spraviť jedinú operáciu. Vymeniť hubu
na prvej pozícii s jeho výstavným kusom. Ostatných húb sa ani nedotkne. Vypíšte, ako bude vyzerať
polička po tejto výmene.

\nadpis{Vstup}

Na prvom riadku je číslo $n$ ($1 \leq n \leq 100000$) -- počet ježkových húb.

Na druhom riadku je $n$ kladných celých čísiel $c_i$ ($1 \leq c_i \leq 1000$) určujúce červivosti
jednotlivých húb v poradí v akom sú na poličke.

\nadpis{Výstup}

Vypíšte, ako vyzerá polička, ak ježko vymení hubu na prvom mieste s jeho výstavným kusom.

\nadpis{Príklady}

\vstup
4
5 7 2 5
\vystup
2 7 5 5
\komentar
Výstavným kusom je huba s červivosťou $2$ na tretej pozície. Po výmene s prvou pozíciou dostaneme
situáciu vo výstupe.
\koniec

\vstup
3
1 2 3
\vystup
1 2 3
\komentar
Výstavný kus už je na prvom mieste, preto sa nič nezmení.
\koniec

\end{document}
