%Tu si môžete zaznačiť, že pracujete na danej veci. V prípade, že ste napísali len časť a ďalej už
%nechcete, alebo ste hotoví tak sa odtiaľ odpíšte. Bolo by však fajn, aby jedu vec robil jeden
%človek ak celok a zvyšný len kontrolovali
%vypracuva: Zaba 
\input ../../include/include.tex

\begin{document}

\velkynadpis{C++ ťahák}

Každý program napísaný v C++ má obsahovať jednu hlavnú funkciu \texttt{main()}. Preto najjednoduchší
program vyzerá nasledovne.

\lstlang{cpp}\begin{lstlisting}
int main(){
}
\end{lstlisting}

Nezabudnite, že každý program musíte pred spustením skompilovať (v DevC++ \texttt{F9}) -- vytvoriť
spustiteľnú binárku a až potom ho spustiť (v DevC++ \texttt{F10}).

\nadpis{Premenné}

Premenná je krabička (miesto v pamäti), kde si program ukladá dáta. Každá premenná má svoj
\texttt{typ} (určuje, aké dáta sa tam dávajú), \texttt{meno} (jednoznačný identifikátor) a
samozrejme obsahuje \texttt{hodnotu} príslušného typu.

\begin{tabular}{| l | l | l | l | l |}
\hline
Typ & Slovný popis & Počet bitov & Rozsah & Poznámka \\ \hline
int & celé čísla & $32$ & $-2^{31}\ ..\ 2^{31}-1$ &  \\ \hline
long long & celé čísla & $64$ & $-2^{63}\ ..\ 2^{63}-1$ &  \\ \hline
char & znak & 8 & & čísla z ASCII tabuľky \\ \hline
bool & pravda/nepravda & $8$ (občas aj $32$ alebo $64$) & \texttt{true}/\texttt{false} & \\ \hline
string & reťazec znakov & & & knižnica \texttt{string} \\ \hline
\end{tabular}

Premenné treba pred použitím zadeklarovať (povedať akého typu je a ako sa volá). Následne pomocou
znaku \texttt{=} vieme do premennej priraďovať hodnotu alebo výraz zložený z matematických
operátorov a iných premenných.

\lstlang{cpp}\begin{lstlisting}
int main() {
    int a, b=4; //tu som si zadeklaroval dve premenne typu int, v b je teraz 4
    a = 4; //do a priradim tiez 4
    b = 2*b+a-3; //do b priradim hodnotu vyrazu napravo
}
\end{lstlisting}

\nadpis{Načítavanie vstupu a výpis výstupu}

Na načítavanie a vypisovanie používame funkcie \texttt{cin} a \texttt{cout}, ktoré patria do
knižnice \texttt{iostream}. Nezabudnite na magický riadok \texttt{using namespace std;}

\lstlang{cpp}\begin{lstlisting}
#include <iostream>
using namespace std;

int main() {
    int a,b;
    //nacitam dve cisla na vstupe, prve do premennej a, druhe do b
    cin >> a >> b;
    //vypisem na vystup cislo b, medzeru, a, text " sucet ", ich sucet a+b a koniec riadka (enter)
    cout << b << " " << a << " sucet " << a+b << endl;
}
\end{lstlisting}

\nadpis{Podmienky}

Ak chceme aby sa program správal inak za splnení istej podmienky, použijeme príkaz \texttt{if}. Ten
zjednodušene vyzerá \texttt{if(<podmienka>) <prikaz>;}. To znamená, \texttt{Ak platí <podmienka>
vykonaj <príkaz>}. \textbf{Nezabudnite}, že porovnávanie dvoch vecí sa vykonáva pomocou \textbf{dvoch}
rovná sa \texttt{==}. Ak chcem vykonať po splnení \texttt{if}u viac príkazov, uzavriem ich do
kučeravých zátvoriek. Takto vyzerá použitie:

\lstlang{cpp}\begin{lstlisting}
int main() {
    int x,p;
    //jednoducha podmienka if
    if(x == 4) p=8;
    //podmienka if s vetvou else
    if(p<10) x=x+2;
    else x=x+14;
    //vetvena podmienka if
    if(x%3 == 0) x=4;
    else if(x%3 == 1) {x=13; p=8;}
    else x=6;
}
\end{lstlisting}

Podmienka vie byť ľubovoľný vyraz, ktorý vieme vyhodnotiť ako \texttt{true} alebo \texttt{false}.
Takisto však vieme skladať viac výrazov pomocou logických spojek \texttt{and} -- \texttt{\&\&} a
\texttt{or} -- \texttt{||}.

Priraď do premennej $x$ číslo $5$, ak $x$ sa rovná $4$ \textbf{alebo} je $p$ rovné $8$ \textbf{x}
rovné $7$.
\lstlang{cpp}\begin{lstlisting}
if(x==4 || (p==8 && x==7)) x=5;
\end{lstlisting}

\end{document}
