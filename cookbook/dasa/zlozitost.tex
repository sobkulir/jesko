%Tu si môžete zaznačiť, že pracujete na danej veci. V prípade, že ste napísali len časť a ďalej už
%nechcete, alebo ste hotoví tak sa odtiaľ odpíšte. Bolo by však fajn, aby jedu vec robil jeden
%človek ak celok a zvyšný len kontrolovali
%vypracuva: Zaba
%
%pre inspiraciu si precitajte toto: http://ksp.mff.cuni.cz/tasks/25/cook1.html
\input ../../include/include.tex

\begin{document}

\velkynadpis{Zložitosť}

Obsah:
\begin{itemize}
    \item cieľ a zámer vyjadrovania zložitosti
    \item $O$-notácia
    \item očakávaná zložitosť od veľkosti vstupu
    \item príklady na notáciu
\end{itemize}

Keď riešime nejaké problémy, môže sa stať, že vymyslím viacero riešení. Ako však povedať, ktoré je
lepšie a vôbec čo to znamená byť lepším. O tom všetkom sa porozprávame práve v tejto kapitole.

Dôležitým aspektom je rýchlosť daného algoritmu. To znamená, ako rýchlo sa pre nejaký vstup doráta k
výsledku. Nechceme predsa čakať na výsledok niekoľko hodín, ak sa dá vyrátať v priebehu sekúnd.
Poďme sa teda zamýšľať nad takzvanou \texttt{časovou zložitosťou} algoritmov.

Hneď na začiatku sa však stretávame s množstvom problémov. Prvým je to, že nechceme porovnávať,
skutočný čas výpočtu. Ten je totiž príliš ovplyvnený počítačom, na ktorom daný program spúšťame.
Nečakáme predsa, že na počítači našej babky bude bežať rovnaký algoritmus rovnako rýchlo ako na
počítači v Googli. Musíme sa teda snažiť oddeliť časovú zložitosť algoritmu od výpočtu na skutočnom
počítači. Bolo by teda fajn, aby sme vedeli odhadnúť  rýchlosť programu z obyčajného pohľadu naň.

Našťastie, to však vôbec nie je také ťažké. Skúste si to sami. Tu vidíte tri programy, ktoré oba
rátajú tú istú vec -- súčet prvých $n$ čísiel.

\listing{program1.cpp}

\listing{program2.cpp}

\listing{program3.cpp}

Myslím, že vám je na prvý pohľad jasné, ktorý z týchto programov je najrýchlejší a naopak, ktorý je
najhorší. Vynásobiť dve čísla je totiž určite jednoduchšie a rýchlejšie, ako pričítavať po
jednotkách k výsledku.



\end{document}
